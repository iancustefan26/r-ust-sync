\documentclass[a4paper,12pt]{article}
\usepackage[utf8]{inputenc}
\usepackage{amsmath, amssymb, amsthm}
\usepackage[utf8]{inputenc}
\usepackage{algorithm}
\usepackage{algpseudocode}

\begin{document}

\section*{Problema 3:}

\subsection*{a)}

\begin{algorithm}
\caption{Determinarea unui subgraf $p$-minimal-cromatic din $G$}
\begin{algorithmic}[1]
\State $p\_min\_crom \gets G$
\While{$\chi(G) == p$}
    \State $p\_min\_crom \gets G$
    \State $G \gets G \setminus v$, $v \in V(G)$
\EndWhile
\State \textbf{return} $p\_min\_crom$
\end{algorithmic}
\end{algorithm}


Algoritmul returnează subgraful $p$-minimal-cromatic generat din $G$. \\

Dacă $G$ nu este minimal cromatic, atunci $\exists v$ a.i. $\chi(G - v) = p$. 
Algoritmul va elimina nodurile $v$ și muchiile lor $e \in E(G)$, 
eventual reducându-se la un subgraf. \\

Algoritmul are finalitate și va merge până la subgraful cu $n = p$ noduri (imposibil de a colora $p$ noduri cu mai puțin de $p$ culori).

\subsection*{b)}
$G$ $p$-minimal-cromatic \\
\textbf{P.P. R.A.} $\delta(G) < p - 1 \Longleftrightarrow \delta(G) \leq p - 2$ \\
$\Longrightarrow \exists v \in V, |N_G(v)| \leq p - 2$. \\

Fie nodul $v$ un nod cu gradul cel mai mic și o $p$-colorare în care nodul $v$ are culoarea $c$. Vecinii lui $v$ sunt colorați în maxim $p-2$ culori, deci $v$ și $N_G(v)$ în maxim $p-1$ culori diferite. Deoarece $G$ este $p$-minimal-cromatic, el este colorat în $p$ culori. \\
$\Longrightarrow \exists$ cel puțin $u \in V$ a.i. $(u, v) \notin E$, pentru care $col(u) \ne col(v)$ și $col(u) \ne col(N_G(v))$. \\
$\Longrightarrow$ Am putea reduce numărul de colorări la $p-1$ colorând pe $v$ în $col(u)$ $\Longrightarrow \chi(G) = p-1$. \\
$\Longrightarrow$ (Absurd) $G$ nu este $p$-minimal-cromatic $\longrightarrow$ contradicție: P.p. inițială este falsă $\Longrightarrow \delta(G) \geq p-1$.

\subsection*{c)}
\textbf{"$\Rightarrow$ (*1)"}\\
\text{$G$ 3-minimal-cromatic}

\begin{enumerate}
    \item 
        \[\delta(G) \geq 2, \quad \max(d_G(v)) < 3.\] \\
        $\implies d_G(v) = 2, \ \forall v \in V \implies G$ circuit indus. \\

    Dacă $G$ este circuit par indus, atunci $\chi(G) = 2$, deoarece se poate colora astfel: \\
    \[
    \text{col}\big(N_G(v)\big) = col(v) * (-1), \quad \text{col}(v) = 1, \ \forall v \in V \implies G \text{ nu este 3-min-crom (absurd).}
    \]
    $\implies G$ circuit impar indus (*1).
\end{enumerate}

\textbf{"$\Leftarrow$ (*2)}

$G$ circuit impar indus. \\
\begin{enumerate}
    \item $G$ 2-min-crom? \\
    \[
    \text{col}(v_i) = \text{col}(N_G(v_{i+1})) \implies \text{col}(v_1) = \text{col}(v_{k+1}),
    \]
    iar cum $G$ este circuit impar indus de forma $(v_1, v_2, \dots, v_{k+1}, v_1) \implies (v_{k+1}, v_1) \in E \implies$ Fals. Nu se poate colora așa.
    
    \item $G$ 3-min-crom? \\
    Adevărat (din ipoteza "$\Leftarrow$").
    
    \item $G$ 4-min-crom? \\
    \[
    \delta(G) \geq 4-1 \text{ (de la b) ) } \implies \delta(G) \geq 3. 
    \]
    Cum $G$ este circuit impar indus $\implies \delta(G) = 2 \implies$ Fals. \\

    $\implies G$ 3-min-crom (*2).
\end{enumerate}
Din (*1) și (*2) $\implies G$ 3-min-crom $\Longleftrightarrow G$ circuit impar indus.

\end{document}

