\documentclass[a4paper,12pt]{article}
\usepackage{amsmath,amsfonts,amssymb}

\begin{document}

\section*{Problema 4}

\subsection*{1.}

Arătăm că \textsc{EDGE\_BIP} $\in \textit{NP}$: Verificarea unui candidat $\Longleftrightarrow$ este echivalentă cu problema \textit{2-col} = $O(n+m)$ $\Longrightarrow \in \textit{NP}$.

\subsection*{2. Definim \textsc{Max-Cut}}
\textbf{Input:} $G = (V, E)$ graf, $c: E  \to \mathbb{R}$ funcție de ponderi, $x \in \mathbb{R}$. \\
\textbf{Output:} Se găsește o tăietură în $G$ de pondere $\geq k$? 

Vom reduce în timp polinomial problema \textsc{MAX-CUT} la \textsc{EDGE\_BIP}, pentru a demonstra că \textsc{EDGE\_BIP} este \textsc{NP}-completă.

Graful rămâne la fel, singura preprocesare:
\[
a = \left(\sum_{e \in E} c(e)\right) - k.
\]
Putem demonstra echivalența dintre cele două prin complementaritatea acestora: \\
Fie 2 mulțimi de noduri $A$ și $B$:
\[
F = \left\{ (i, j) \in E \mid i \in A, j \in B \right\} = \text{tăietura},
\]
\[
M = \left\{ (i, j) \in E \mid i \in A \text{ și } j \in A \text{ sau } i \in B \text{ și } j \in B \right\}.
\]
Putem observa că: \\
1) $F \cup M = E$ și $F \cap M = \emptyset$, partiție a muchiilor. \\
2) Prin ștergerea tuturor muchiilor din $M$, graful devine bipartit $(A, B, E)$. \\
$\Longrightarrow M$ este mulțimea cerută de \textsc{EDGE-BIP}.

Din $1)$:
\[
\sum_{e \in F} c(e) + \sum_{e \in M} c(e) = \sum_{e \in E} c(e)
\]
\[
\text{$"=>"$ Dacă ponderea lui } F \geq k:
\]
\[
\sum_{e \in E} c(e) - \sum_{e \in M} c(e) \geq k
\]
\[
\implies \sum_{e \in M} c(e) \leq \sum_{e \in E} c(e) - k (= a)
\]
\[
\text{$"<="$ Dacă ponderea lui } M \leq \sum_{e \in E} c(e) - k:
\]
\[
\implies \text{(similar)} \sum_{e \in F} c(e) \geq k
\]

Am demonstrat că \textsc{MAX-CUT} se reduce polinomial la \textsc{EDGE\_BIP}.

\end{document}

